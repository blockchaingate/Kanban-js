% % % % % % % %
% % % % % % % %
%IMPORTANT
%compiles with
%pdflatex -shell-escape
%IMPORTANT
% % % % % % % %
% % % % % % % %
%On windows: comment out the line \input{../system-specific-config-Ubuntu-texlive}
\documentclass%
%[handout]%
{beamer}
%IMPORTANT: the following line selects the current lecture. Change number to select a different lecture.

\newcommand{\currentLecture}{1}

\mode<presentation>
{
\useinnertheme{rounded}
\useoutertheme{infolines}
\usecolortheme{orchid}
\usecolortheme{whale}
}
%\setbeamertemplate{footline}{%
%  \raisebox{5pt}{\makebox[\paperwidth]{\hfill\makebox[10pt]{\scriptsize\insertframenumber}}}}

% Or whatever. Note that the encoding and the font should match. If T1
% does not look nice, try deleting the line with the fontenc.

\graphicspath{{../../modules/}}

\newcommand{\lect}[4]{
\ifnum#3=\currentLecture
  \date{#1}
  \lecture[#1]{#2}{#3}
#4
\else
%include nothing
\fi
}

\setbeamertemplate{footline}
{
  \leavevmode%
  \hbox{%
  \begin{beamercolorbox}[wd=.333333\paperwidth,ht=2.25ex,dp=1ex,center]{author in head/foot}%
    \usebeamerfont{author in head/foot}\insertshortauthor
  \end{beamercolorbox}%
  \begin{beamercolorbox}[wd=.333333\paperwidth,ht=2.25ex,dp=1ex,center]{title in head/foot}%
    \usebeamerfont{title in head/foot}\insertshorttitle
  \end{beamercolorbox}%
  \begin{beamercolorbox}[wd=.333333\paperwidth,ht=2.25ex,dp=1ex,center]{date in head/foot}%
    \usebeamerfont{date in head/foot}\insertshortdate{}
  \end{beamercolorbox}}%
  \vskip0pt%
}

\setbeamertemplate{navigation symbols}{}



% If you have a file called "university-logo-filename.xxx", where xxx
% is a graphic format that can be processed by latex or pdflatex,
% resp., then you can add a logo as follows:

%\pgfdeclareimage[height=0.8cm]{logo}{bluelogo}
%\logo{\pgfuseimage{logo}}

\begin{document}

\AtBeginLecture{%

\title[\insertlecture]{\hspace{-1.3cm} {\raisebox{-0.7cm}[0cm][0cm]{ \includegraphics[height =1.2cm]{../../html/fabcoin.png}}}~~~~Kanban~~~~~~~}
\subtitle{\insertlecture}
\author[\raisebox{-0.115cm}{\includegraphics[height=0.4cm]{../../html/fabcoin.png}} Kanban]{ Todor Milev, Ph.D.
}
\institute[FAB]{Senior C++ programmer \\ FA Enterprise System}
\date{\insertshortlecture}
\begin{frame}
  \titlepage
\end{frame}

\begin{frame}{Outline}
  \tableofcontents[pausesections]
\end{frame}
}%
\lect{May 18}{Progress report}{1}{
\section{Current architecture plan}
\begin{frame}
\frametitle{Current architecture plan}
\begin{itemize}
\item OS: Ubuntu at the moment. Aspirations to support everything.
\item Computational engine in C/C++.
\begin{itemize}
\item Crypto functions in openCL/C++.
\item Data management? Expected solution: stl library (no database).
\end{itemize}
\item Networking, management, testing, other non-computational tasks: nodejs.
\end{itemize}
\end{frame}
\subsection{Computational engine}
\begin{frame}
\frametitle{Computational engine}
\begin{itemize}
\item Core cryptographic functions: same code compiled as both openCL/C++.
\begin{itemize}
\item Forked Pieter Wuille's secp256k1 library (used in fabcoin \& bitcoin).
\item Heavy but formal modifications: more than 10\% new/modified LOC.
\begin{itemize}
\item openCL does not 
\end{itemize}
\end{itemize}
\end{itemize}
\end{frame}
}
\end{document}