% % % % % % % %
% % % % % % % %
%IMPORTANT
%compiles with
%pdflatex -shell-escape
%IMPORTANT
% % % % % % % %
% % % % % % % %
% !TeX spellcheck = en_US
%On windows: comment out the line \input{../system-specific-config-Ubuntu-texlive}
\documentclass%
%[handout]%
{beamer}
%IMPORTANT: the following line selects the current lecture. Change number to select a different lecture.

\newcommand{\currentLecture}{1}

\mode<presentation>
{
\useinnertheme{rounded}
\useoutertheme{infolines}
\usecolortheme{orchid}
\usecolortheme{whale}
}
%\setbeamertemplate{footline}{%
%  \raisebox{5pt}{\makebox[\paperwidth]{\hfill\makebox[10pt]{\scriptsize\insertframenumber}}}}

% Or whatever. Note that the encoding and the font should match. If T1
% does not look nice, try deleting the line with the fontenc.

\graphicspath{{../../modules/}}

\newcommand{\lect}[4]{
\ifnum#3=\currentLecture
  \date{#1}
  \lecture[#1]{#2}{#3}
#4
\else
%include nothing
\fi
}

\setbeamertemplate{footline}
{
  \leavevmode%
  \hbox{%
  \begin{beamercolorbox}[wd=.333333\paperwidth,ht=2.25ex,dp=1ex,center]{author in head/foot}%
    \usebeamerfont{author in head/foot}\insertshortauthor
  \end{beamercolorbox}%
  \begin{beamercolorbox}[wd=.333333\paperwidth,ht=2.25ex,dp=1ex,center]{title in head/foot}%
    \usebeamerfont{title in head/foot}\insertshorttitle
  \end{beamercolorbox}%
  \begin{beamercolorbox}[wd=.333333\paperwidth,ht=2.25ex,dp=1ex,center]{date in head/foot}%
    \usebeamerfont{date in head/foot}\insertshortdate{}
  \end{beamercolorbox}}%
  \vskip0pt%
}

\setbeamertemplate{navigation symbols}{}



% If you have a file called "university-logo-filename.xxx", where xxx
% is a graphic format that can be processed by latex or pdflatex,
% resp., then you can add a logo as follows:

%\pgfdeclareimage[height=0.8cm]{logo}{bluelogo}
%\logo{\pgfuseimage{logo}}


\usepackage[latin1]{inputenc}
\usepackage{etex}
\usepackage{ifthen}
\usepackage[strings]{underscore}
\usepackage{tikz}
\usetikzlibrary{calc}
\usepackage{bbding}
\let\Cross\relax
\let\Square\relax
\usepackage{amsmath}
\usepackage{amssymb}
\usepackage{cancel}
\usepackage{comment}
\usepackage{multirow}
\usepackage{multicol}
\usepackage{psfrag}
\usepackage{rotating}
\usepackage{fp}
\usepackage{calc}
\usepackage{bm}
\usepackage[all,cmtip]{xy}
\RequirePackage{xstring}
\usepackage{times}
\usepackage[english]{babel}
\usepackage{longtable}
\usepackage{graphicx}
\usepackage{verbatim}
\usepackage{array}
\usepackage[breakwords]{truncate}



\newcommand{\alertNoH}[2]{\alert<handout:0|#1>{#2}}

\newcommand{\fcAnswerNoH}[2]{
\FPeval{\fcResult}{clip(#1-1)}
\uncover<handout:0|\fcResult>{\alertNoH{\fcResult}{\textbf{?} }} \worksheet{\uncover<handout:0| #1->{\alertNoH{#1}{\!\!\!#2}}}
}
\newcommand{\fcAnswer}[2]{%
\uncover<handout:0|\the\numexpr#1-1\relax>{\alertNoH{\the\numexpr#1-1\relax}{\textbf{?}}}\worksheet{\uncover<#1->{\alertNoH{#1}{\!\!\!#2}}}%
}%

\newcommand{\fcAnswerUncover}[3]{%
\FPeval{\fcResult}{clip(#2-1)}%
\uncover<handout:0|#1-\fcResult>{\alertNoH{\fcResult}{\textbf{?}}}\worksheet{\uncover<#2->{\alertNoH{#2}{\!\!\!#3}}}
%\makebox[\widthof{#3}][c]{\only<handout:0|#1-\fcResult>{\alertNoH{\fcResult}{\textbf{?}}} \only<#2->{\alertNoH{#2}{\!\!\!#3}}}%
}
\newcommand{\fcAnswerUncoverNew}[4]{%
\uncover<handout:0|.(#1)-.(#2)>{\alertNoH{.(#2)}{\textbf{?}}}\worksheet{\uncover<.(#3)->{\alertNoH{.(#3)}{\!\!\!#4}}}
%\makebox[\widthof{#3}][c]{\only<handout:0|#1-\fcResult>{\alertNoH{\fcResult}{\textbf{?}}} \only<#2->{\alertNoH{#2}{\!\!\!#3}}}%
}

\newcommand{\fcAnswerUncoverNoH}[3]{
\FPeval{\fcResult}{clip(#2-1)}
\uncover<handout:0|#1-\fcResult>{\alertNoH{\fcResult}{\textbf{?}}}\worksheet{\uncover<handout:0|#2->{\alertNoH{#2}{\!\!\!#3}}}
}

\newcommand{\fcQuestion}[2]{%
\FPeval{\fcResult}{clip(#1+1)}%
\uncover<#1->{\alertNoH{ #1,\fcResult}{#2}}%
}
\newcommand{\fcEvalToInt}[1]{\FPeval{\fcResult}{clip(#1)}\fcResult}
\newcommand{\refBad}[3]{%
\ifthenelse{\equal{#1}{??}}%
{#2}%
{#3}%
}%example usage: \refBad{\ref{eqMacLaurinDef}}{their definition}{their definition (Definition \ref{eqMacLaurinDef})}
\newcommand{\fcCancel}[2]{%
\FPeval{\fcResult}{clip(#1-1)}%
\only<handout:0|-\fcResult>{{#2}} \only<#1->{{\alertNoH{#1}{{\cancel{{\alertNoH{0}{{#2}}}}}}}}
\vphantom{{\cancel{#2}}}%
}



\begin{document}

\AtBeginLecture{%

\title[\insertlecture]{\hspace{-1.3cm} {\raisebox{-0.7cm}[0cm][0cm]{ \includegraphics[height =1.2cm]{../../html/fabcoin.png}}}~~~~Bootstrap kanbanGO~~~~~~~}
\subtitle{\insertlecture}
\author[\raisebox{-0.115cm}{\includegraphics[height=0.4cm]{../../html/fabcoin.png}} Kanban prototype]{ Todor Milev, Ph.D.
}
\institute[FAB]{Senior C++ programmer \\ FA Enterprise System}
\date{\insertshortlecture}
\begin{frame}
  \titlepage
\end{frame}

\begin{frame}{Outline}
  \tableofcontents[pausesections]
\end{frame}
}%
\lect{\today}{Progress December 21}{1}{
\begin{frame}
\begin{itemize}
\item Node.js. 
\begin{itemize}
\item Work done.
\begin{itemize}
\item Fixed memory leak in server.
\item Fixed bug: fabcoind logs were not showing correctly.
\item Added monitoring infrastructure to help prevent a future leak.
\item Log files.
\item Reworked transaction builder user inteface: fabcoind and kanbanGO.
\end{itemize}
\item To do.
\begin{itemize}
\item Create an ssh tunnel user interface.
\end{itemize}
\end{itemize}
\item KanbanGO: work done.
\begin{itemize}
\item Fixed a flexible transaction format for SCAR write operations. New coinbase-like transaction \& new opcode.
\item Transaction builder now signs the input transactions.
\item KanbabGO $SHA256^{\circ 2}$ ECDSA implementation. 
\item KanbanGO: one private key now supports Schnorr, ECDSA $SHA256^{\circ 2}$ and ECDSA keccak signatures.
\item RPC calls for testing the above.
\end{itemize}
\item KanbanGO: to do.
\begin{itemize}
\item Ensure transaction built in KanbanGO passes through fabcoind.
\end{itemize}
\end{itemize}
\end{frame}

\begin{frame}[fragile]
\frametitle{Fabcoind}
\begin{itemize}
\item Work done.
\begin{itemize}
\item New function \verb|CScript::FitsOpCodePattern|.
\item Added human-readable reports to the internals of the system.
\item Bug fixes in debug-session logs. 
\item New opcode : \verb|OP_CONTRACTCOVERSFEES|.
\item New transaction type \verb|contract_covers_fees|. 
\item New coinbase-like transaction txid.
\end{itemize}
\item To do.
\begin{itemize}
\item Insert code to accept \verb|OP_CONTRACTCOVERSFEES|. 
\item Insert code to accept the new coinbase-like transaction txid.
\item Both of these require modifications of:
\begin{itemize}
\item Memory pool acceptance code.
\item Block verification code.
\item ... and possibly more?
\end{itemize}
\end{itemize}
\end{itemize}

\end{frame}
}

\end{document}